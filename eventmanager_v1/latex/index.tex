The documentation for basic modules for a simple analysis toolkit based on a input data format in {\tt }\mbox{[}ROOT\mbox{]}(\href{http://root.cern.ch}\tt{http://root.cern.ch}) and an internal event data model employing \mbox{[}Fastjet\mbox{]}(\href{http://www.fastjet.fr}\tt{http://www.fastjet.fr}) is available in this help system.

The toolkit features event analysis for (1) individual (single interaction) signal events and (2) pile-up events with many interactions. All software supporting the merging with client defined pile-up conditions is included.\hypertarget{index_data_format}{}\section{Supported data format}\label{index_data_format}
Both signal and pile-up events are expected to be in a simple {\tt ROOT} tuple structure providing the following variables: 

\begin{Code}\begin{verbatim} #define _FLOAT_T_ float 
 int Nentry;                  // total # entries
 int Npartons;                // # partons
 int Nparticles;              // # particles
 int ID[Nentry];              // PDG Id
 int Stat[Nentry];            // internal status word (-1,-2 for partons, 2 for particles)
 _FLOAT_T_ Charge[Nentry];    // charge
 _FLOAT_T_ Px[Nentry];        // momentum Px
 _FLOAT_T_ Py[Nentry];        // momentum Py
 _FLOAT_T_ Pz[Nentry];        // momentum Pz
 _FLOAT_T_ P0[Nentry];        // energy E
 _FLOAT_T_ Pm[Nentry];        // mass m
 _FLOAT_T_ Pt[Nentry];        // transverse momentum
 _FLOAT_T_ Rap[Nentry];       // rapidity y
 _FLOAT_T_ Phi[Nentry];       // azimuth phi
 _FLOAT_T_ Eta[Nentry];       // pseudo-rapidity eta
\end{verbatim}\end{Code}



Presently the toolkit does not support dynamic allocation of branches (flexible branch names). Parton kinematics is stored in the first {\tt Npartons} entries (array indices {\tt 0}...{\tt Npartons-1}), while stable particle kinematics is stored at indices {\tt Npartons}...{\tt Nentry-1}. Thus, {\tt Nparticles=Nentry-Npartons}.\hypertarget{index_event_data_model}{}\section{Event data model}\label{index_event_data_model}
The event data model is based on {\tt fastjet::Pseudo\-Jet} (Fastjet versions 3.x required). Each {\tt Pseudo\-Jet} represents a particle or parton. A user info object {\tt Particle\-Info} is attached which stores particle ID, charge, vertex association, and other useful characteristics not accommodated by {\tt Pseudo\-Jet} directly.

The {\tt \hyperlink{classEvent}{Event}} container typically holds all partons, particles, and vertices characterizing a given event. Helper functions are provided which fill the {\tt \hyperlink{classEvent}{Event}} structure.

\begin{Desc}
\item[Note:]{\tt \hyperlink{classEvent}{Event}} does not control its data content. It needs to be reset by the client before a new event is loaded. Otherwise it will add final state partons and particles to the previous event each time the corresponding data handler ({\tt \hyperlink{structDataHandler}{Data\-Handler}}) method is invoked.\end{Desc}
\hypertarget{index_merging_strategy}{}\section{Event merging}\label{index_merging_strategy}
\begin{Desc}
\item[\hyperlink{todo__todo000004}{Todo}]There is no unpacking of partons available right now. Events are only available in the stable particle representation. \end{Desc}
